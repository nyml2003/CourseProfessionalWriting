\chapter{研究进展}

\section{输入提升}
输入指的是用户的查询,该查询最初被输入到检索器中。输入的质量显著影响检索阶段的最终结果,因此对输入优化变得至关重要。在这里,我们将介绍两种方法:查询改写和数据增强。
查询改写可以通过修改输入查询来提高检索结果。数据增强是指在检索之前提前对数据进行改进,如去除无关信息、消除歧义、更新过时文档、合成新数据等,可以有效提高最终 RAG 系统的性能
\cite{zhao2024retrievalaugmentedgenerationaigeneratedcontent}。
\section{检索器优化}
在 RAG 系统中,检索过程对结果影响很大。一般来说,内容质量越好,就越容易激发 LLM 的上下文学习能力以及其他生成模型的能力;内容质量越差,就越有可能导致模型幻觉。常见的优化方法有以下几种:

递归检索是一种高级的检索策略,它通过在检索之前拆分查询,并执行多次搜索以检索更多、更高质量的内容。这种策略的核心思想是在不同层次上构建chunks节点与检索器,并建立层次之间的链接关系,使得能够在每次检索时自动实现向下递归探索,直至达到结束条件。

检索器微调是对检索器的优化,一般是对嵌入模型能力的提升。检索器的能力越强,就可以为后续生成器提供更多有用的信息,从而提高 RAG 系统的有效性。一个好的嵌入模型可以使语义相似的内容在向量空间中更紧密地结合在一起;此外,对于已经具有良好表达能力的嵌入模型,我们仍然可以使用高质量的领域数据或任务相关数据对其进行微调,以提高其在特定领域或任务中的性能。

相比单检索方式,混合检索更具全面性,混合检索是指同时使用多种类型的检索器,如同时使用统计词频的方式和计算向量相似性的方式来得到检索结果,也就是混合了稀疏检索和密集检索。稀疏检索侧重于关键词的精确匹配,适用于搜索特定术语,如产品名称或专业术语。而密集检索则侧重于理解查询和文档的上下文和含义,适用于捕捉语义相似性,即使查询中不存在确切的关键字也能检索到相关信息。因此,混合检索对不同类型的查询更具鲁棒性,无论它们是精确的基于关键字的查询,还是更抽象且依赖于上下文的查询。
例如,Hybrid with HyDE \cite{wang2024searchingbestpracticesretrievalaugmented} 方法将稀疏和稠密检索结合起来,从语义和语义角度捕捉相关文档。

此外,加入重排序技术,对检索到的内容进行重新排序,可以实现更大的多样性和更好的结果。

\section{语言模型的改进}
在 RAG 系统中,生成器的质量通常决定最终输出结果的质量。在这里,我们将介绍如下一些提升生成器能力的技术。
提示词工程是一种专注于提高 LLM 输出质量的技术,其中包括提示词压缩、回退提示、主动提示、思维链提示等等,以上这些同时也都适用于使用 LLM 生成器的 RAG 系统中。
解码过程控制、调整是指在生成器处理过程中添加额外的控制,可以通过调整超参数来实现更大的多样性或者以某种形式限制输出词汇表等等。
生成器微调可以使生成模型具有更精确的领域知识或更好地与检索器匹配的能力。
\section{RAG 流程提升}
我们将对整个 RAG 流程上的优化分为如下两大类:自适应检索和迭代 RAG。

自适应检索是基于一个观察:很多 RAG 的研究和实践表明,检索并不总是有利于最终生成的结果。当模型本身的参数化知识足以回答相关问题时,过度检索会造成资源浪费,并可能增加模型的混乱。因此,一些工作提出了基于规则和基于模型的自适应检索方法。
基于规则等方法指的是通过判断某些与模型生成高度相关的指标来确定是否进行搜索,具体而言,这个变量可以是模型生成过程中当前 token 的生成概率,也可以是模型的困惑度等等。基于模型的方法则指的是借助模型能力来判断是否进行搜索,这里的模型可以是生成模型本身也可以是借助外部模型。

迭代RAG则指的是迭代的进行检索和生成。生成器的当前轮次输出可以在一定程度上反映其仍然缺乏的知识,并且检索器可以检索缺失的信息作为下一轮的上下文信息,这有助于提高下一轮生成内容的质量。如此循环迭代,直到生成内容达到标准。