\chapter{引\hspace{6pt}言}

\section{背景介绍}

这是一个测试文本,用于模拟真实内容的效果。它可以帮助设计师和开发者在没有实际内容的情况下进行排版和布局测试。这个文本没有实际意义,只是为了展示字体、段落和页面布局的效果。
Lorem ipsum dolor sit amet, consectetur adipiscing elit, sed do eiusmod tempor incididunt ut labore et dolore magna aliqua. Ut enim ad minim veniam, quis nostrud exercitation ullamco laboris nisi ut aliquip ex ea commodo consequat.

计算电磁学方法\citing{wang1999sanwei, liuxf2006, zhu1973wulixue, chen2001hao, gu2012lao, feng997he}从时、频域角度划分可以分为频域方法与时域方法两大类。频域方法的研究开展较早,目前应用广泛的包括:矩量法(MOM)\citing{xiao2012yi,zhong1994zhong}及其快速算法多层快速多极子(MLFMA)\citing{clerc2010discrete}方法、有限元(FEM)\citing{wang1999sanwei,zhu1973wulixue}方法、自适应积分(AIM)\citing{gu2012lao}方法等,这些方法是目前计算电磁学商用软件
\footnote{脚注序号“\ding{172},……,\ding{180}”的字体是“正文”,不是“上标”,序号与脚注内容文字之间空1个半角字符,脚注的段落格式为:单倍行距,段前空0磅,段后空0磅,悬挂缩进1.5字符;中文用宋体,字号为小五号,英文和数字用Times New Roman字体,字号为9磅;中英文混排时,所有标点符号(例如逗号“,”、括号“()”等)一律使用中文输入状态下的标点符号,但小数点采用英文状态下的样式“.”。}
(例如:FEKO、Ansys 等)的核心算法。由文献\cite{feng997he,clerc2010discrete,xiao2012yi}可知

\subsection{大语言模型(LLMs)的兴起}
这是一个测试文本,用于模拟真实内容的效果。它可以帮助设计师和开发者在没有实际内容的情况下进行排版和布局测试。这个文本没有实际意义,只是为了展示字体、段落和页面布局的效果。

\subsection{LLMs的局限性}
这是一个测试文本,用于模拟真实内容的效果。它可以帮助设计师和开发者在没有实际内容的情况下进行排版和布局测试。这个文本没有实际意义,只是为了展示字体、段落和页面布局的效果。

\section{检索增强生成(RAG)技术的诞生}
这是一个测试文本,用于模拟真实内容的效果。它可以帮助设计师和开发者在没有实际内容的情况下进行排版和布局测试。这个文本没有实际意义,只是为了展示字体、段落和页面布局的效果。

\section{RALMs的研究意义和应用前景}
这是一个测试文本,用于模拟真实内容的效果。它可以帮助设计师和开发者在没有实际内容的情况下进行排版和布局测试。这个文本没有实际意义,只是为了展示字体、段落和页面布局的效果。
