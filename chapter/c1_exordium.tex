\chapter{引\hspace{6pt}言}

\section{背景介绍}
近年来,大规模预训练语言模型(Large Language Models, LLMs)在自然语言处理(NLP)领域取得了显著突破\cite{vaswani2023attentionneed}。以 BERT 和 GPT 系列为代表的模型,通过海量数据的预训练,展现了强大的语言生成和理解能力。这些 LLMs 被广泛应用于文本生成、机器翻译和对话系统等多个任务,推动了智能应用的普及\cite{devlin2019bertpretrainingdeepbidirectional}\cite{brown2020languagemodelsfewshotlearners}。
然而,尽管 LLMs 具备强大的语言处理能力,仍存在明显的局限性。首先,LLMs 的知识来源于其训练数据,无法动态更新。这导致它们在处理实时信息或领域专有知识时,可能生成不准确或过时的答案。此外,模型有时会出现“幻觉”(hallucination),即生成与事实不符的内容。这些局限性表明,传统 LLMs 在处理知识密集型任务时存在瓶颈,无法满足某些高精度任务的要求\cite{Ji_2023}。
为了解决这些问题,研究者提出了检索增强生成(Retrieval-Augmented Generation, RAG)技术。这种技术通过将语言模型与信息检索模块结合,使模型不仅依赖于预训练的内部知识,还能够动态检索外部信息,从而生成更准确、时效性更高的内容。RAG 架构不仅弥补了 LLMs 在知识覆盖和实时性上的不足,还显著减少了幻觉现象,使得生成结果更加可靠\cite{lewis2021retrievalaugmentedgenerationknowledgeintensivenlp}\cite{chan2024rqraglearningrefinequeries}。
根据《RAG+RAU:对检索增强型语言模型(RALM)进行全面、深入综述》的分析,RAG 技术使得语言模型能够在生成过程中从外部知识库中检索相关信息,从而改善其生成能力\cite{hu2024ragrausurveyretrievalaugmented}。这种创新在问答系统、对话生成等信息密集型任务中取得了显著成效,推动了语言生成技术的新一轮发展。
\section{RALMs 的研究意义与应用前景}
检索增强大语言模型(Retrieval-Augmented Large Language Models, RALMs)是 RAG 技术的进一步扩展,通过实时检索外部信息提升语言模型的知识覆盖面和准确性。与传统 LLMs 相比,RALMs 不再仅依赖其内置的语言模式,而是通过检索模块获取最新、最相关的外部知识来生成更为准确的内容。
这一创新在多个应用领域展现出广泛的前景,尤其是在问答系统、知识推理、医疗诊断等知识密集型任务中。RALMs 的优势在于,它们可以通过检索模块实时获取信息,而不是仅仅依赖训练期间学习到的知识。这种能力使它们在复杂、动态的任务环境中表现得尤为出色。
此外,检索增强技术还有效降低了模型对庞大参数规模的依赖,从而减少了模型的计算成本。通过引入检索机制,RALMs 可以更高效地处理长文本和复杂推理任务,这在领域专有知识和实时数据检索场景中尤为重要。