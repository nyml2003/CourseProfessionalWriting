\chapter{RAG技术详解}
\section{RAG的核心范式}
检索增强型语言模型(RAG)是一种结合了检索(Retrieval)和生成(Generation)的先进自然语言处理技术。RAG的核心在于利用外部信息源来增强大型语言模型(LLMs)的性能,以解决传统模型在处理特定领域知识时遇到的挑战,如幻觉问题和领域知识的缺失。RAG技术可以分为两个主要范式:检索增强生成(RAG)和检索增强理解(RAU)。
\subsection{检索增强生成(RAG)}
检索增强生成(RAG)主要关注于利用检索到的信息来辅助生成任务,如文本生成、机器翻译和对话系统。在这一范式中,模型首先通过检索器从大量数据中检索出与输入相关的信息,然后将这些信息作为上下文输入到语言模型中,以生成流畅、准确且信息丰富的文本。
\subsection{检索增强理解(RAU)}
检索增强理解(RAU)则侧重于利用检索信息来提升模型对文本的理解能力,这在问答系统\cite{Ahn2022}、文本分类和事实核查等任务中尤为重要。RAU通过检索相关信息来增强模型对输入文本的语义理解,从而提高任务的准确性。
\section{RAG的组成要素}
RAG系统的架构主要由两个部分组成:检索器、语言模型。
\subsection{检索器}
检索器负责从大量的数据源中检索出与输入查询最相关的信息。检索器的性能直接影响到RAG系统的效果,因此,设计高效准确的检索器是RAG技术的关键。
\subsection{语言模型}
语言模型是RAG系统的另一个核心组成部分,它负责生成或理解文本。根据任务的不同,可以选择不同类型的语言模型,如自编码器模型、自回归模型或编码器-解码器模型。
\section{RAG的关键技术}
\subsection{检索器类型}
检索器在RALM架构中扮演着至关重要的角色。通过检索器获取的信息可以显著提高大语言模型的准确性。
\subsubsection{稀疏检索}
TF-IDF算法是一种统计方法,用以评估一个词语对于一个文件集或一个语料库中的其中一份文件的重要性。

词频(TF)\cite{KarenSparckJones}表示词在文档中出现的次数,计算公式通常是:

\begin{equation}
\text{TF}(t,d) = \frac{f_{t,d}}{\sum_{t' \in d} f_{t',d}}
\end{equation}

其中: 
$TF(t,d)$ 是词t在文档d中的词频。
$f_{t,d}$ 是词t在文档d中出现的频率。
$\sum_{t' \in d} f_{t',d}$ 是文档d中所有词的频率之和。

逆文档频率(IDF)表示词在整个语料库中出现的频率的倒数,计算公式通常是:

\begin{equation}
\text{IDF}(t,D) = \log\left(\frac{|D|}{|\{d \in D : t \in d\}|}\right)
\end{equation}

其中:
$IDF(t,D)$ 是词t的逆文档频率。
$|D|$ 是语料库中文档的总数。
$|\{d \in D : t \in d\}|$ 是包含词t的文档数量。

TF-IDF是TF和IDF的乘积,用于衡量词在文档中的重要性,计算公式通常是:

\begin{equation}
\text{TF-IDF}(t,d,D) = \text{TF}(t,d) \times \text{IDF}(t,D)
\end{equation}

BM25算法\cite{robertson1995okapi}是基于TF和IDF的基础上进行改进的一种检索算法,它考虑了词频、文档长度和文档集合中词的分布,计算公式通常是:

\begin{equation}
\text{BM25}(q,d) = \sum_{i=1}^{n} \text{IDF}(q_i) \times \frac{f(q_i, d) \times (k_1 + 1)}{f(q_i, d) + k_1 \times (1 - b + b \times \frac{|d|}{\text{avgdl}})}
\end{equation}

其中:
$\text{BM25}(q,d)$ 是查询q和文档d之间的BM25得分。
$\text{IDF}(q_i)$ 是查询词$q_i$的逆文档频率。
$f(q_i, d)$ 是查询词$q_i$在文档d中的词频。
$k_1$ 和 $b$ 是BM25算法的调节参数。
$|d|$ 是文档d的长度。
$\text{avgdl}$ 是文档集合的平均长度。

对于每个查询词,BM25算法会计算它在每个文档中的得分,然后将这些得分相加,得到该文档对于整个查询的总得分。在BM25的上下文中,所说的“稀疏向量”通常指的是文档的向量表示,其中只包含非零项,即那些查询词在文档中出现的词频和逆文档频率的乘积。由于大多数词在大多数文档中不会出现,所以这种表示通常是稀疏的。最后,通过比较查询向量和文档向量之间的相似性(例如,使用余弦相似度),可以找到与查询最相关的文档。

稀疏检索最初依赖于匹配相关内容的方法,如TF-IDF(Term Frequency-Inverse Document Frequency)和BM25算法。这些算法通过计算词频和逆文档频率来评估相关性,具有简单和快速的优点。随着机器学习技术的发展,稀疏向量被用来表示词,并通过网络距离计算来检索它们。稀疏检索在RALMs中可以用于多种任务,包括自动翻译、文本分类、情感分析等。它特别适用于那些基于知识的任务,因为这些任务通常需要从大量文档中检索信息。

稀疏检索通常依赖于倒排索引和原始数据输入。这种方法简单、不依赖于训练,但性能受限于数据库质量和查询。

\subsubsection{密集检索}
密集检索使用深度学习技术来生成查询和文档的稠密向量表示,然后通过计算向量之间的距离来检索信息。这种方法能够更好地捕捉到查询和文档之间的语义关系,但计算成本较高。

在密集检索中,常用的架构是双编码器(Dual-Encoder)模型,它包含两个独立的网络,分别对查询和文档进行编码,然后通过计算编码向量之间的相似度来检索相关信息。

在密集检索中,词嵌入是一种常见的方法,它使用深度学习技术将词映射到高维向量空间。这些嵌入能够捕捉词之间的语义关系,从而提高检索的准确性。例如,DPR(Dense Passage Retriever)模型\cite{karpukhin2020densepassageretrievalopendomain}就是一种使用密集嵌入的检索模型,它通过在低维连续空间中索引所有段落,使得在运行时高效地检索与输入问题相关的前k个段落成为可能

\subsubsection{互联网检索}
互联网检索是指直接从互联网上检索信息的方法。这种方法可以获取到最新的信息,但面临的挑战包括信息的准确性和相关性。互联网检索可以作为稀疏检索和密集检索的补充,以提高检索的效果。搜索引擎本身利用了大量的数据和传统检索方法,可以作为RALMs的一个重要组成部分,增强模型的时效性和泛化能力。\cite{komeili2021internetaugmenteddialoguegeneration}
\subsubsection{混合检索}
混合检索结合了多种检索技术,以提高检索的准确性和鲁棒性。这种方法可以结合稀疏检索的效率和密集检索的语义理解能力,或者结合互联网检索的最新信息。\cite{lazaridou2022internetaugmentedlanguagemodelsfewshot}\cite{boytsov2016off}
\subsection{语言模型的选择}
与仅依靠训练参数完成任务的传统语言模型不同,RAG 中的语言模型通过整合检索器获取的非参数记忆和自身的参数记忆,形成半参数记忆,从而增强了语言模型的性能。
\subsubsection{自编码器语言模型}
自编码器语言模型,如BERT,通过预测遮蔽词来学习语言的表示。BERT采用了一种遮蔽语言模型(Masked Language Model,MLM)的训练方式,即在输入文本中,随机选择一些单词(通常是 15\% 左右)并将其替换为特殊的 [MASK] 标记。模型的任务是预测这些被遮蔽的单词。这种方式迫使模型学习单词之间的双向关系,因为它需要考虑整个句子的上下文来预测被遮蔽的单词,而不仅仅是单向的前文或后文信息。这类模型在理解任务中表现出色,因为它们能够捕捉到丰富的上下文信息。

自编码器语言模型常用于自然语言理解(NLU)任务,如在一些 RALM 架构中用于判断等特定任务。具有高度的泛化能力,是无监督学习的,不需要数据标注,能够自然地融入上下文语义信息。
\subsubsection{自回归语言模型}
自回归语言模型,如GPT系列,通过预测下一个词来生成文本。GPT 采用了无监督预训练的方式,在大规模的文本语料上进行学习。它以预测下一个单词为目标,给定一段文本中的前几个单词,模型尝试预测下一个单词是什么。

自回归语言模型适用于自然语言生成(NLG)任务,如对话生成和机器翻译等,是 RAG 中处理 NLG 任务的流行选择。采用从左到右的语言建模方式,能够根据前面的单词预测下一个单词,适合生成式自然语言处理任务。
\subsubsection{编码器-解码器模型}
编码器-解码器模型,如T5\cite{raffel2023exploringlimitstransferlearning},结合了编码器和解码器的结构,使其在处理需要转换输入到输出的任务时非常有效,如机器翻译。

T5 引入了一个统一的框架,将所有基于文本的语言问题转化为文本到文本的格式。这使得它在 RAG 中能够更方便地处理各种自然语言处理任务,无论是文本生成、问答还是其他任务。例如,在检索增强的问答任务中,它可以将问题和检索到的相关文档作为输入,以生成准确的答案。

\subsection{检索粒度}
检索粒度指的是检索单元的索引级别,如文档、段落、令牌或其他级别(如实体)\cite{chen2024densexretrievalretrieval}\cite{lee2021phraseretrievallearnspassage}。
文档级检索:将整个文档作为检索单元。
段落级检索(Chunk Retrieval):在传统和基于LLM的RAG模型中较为常见,如REALM、RAG和Atlas。
令牌级检索(Token Retrieval):更细粒度的检索,适用于需要罕见模式或领域外数据的情况。
实体检索(Entity Retrieval):从知识而非语言的角度设计,如Entities as Experts (EAE)模型,它通过实体记忆来表示知识。
\subsection{生成器设计}

参数可访问生成器\cite{2022Compare}(White-box Generators):如Encoder-Decoder和Decoder-only模型,允许参数优化,可以针对不同的检索和增强方法进行训练以提高生成性能。
参数不可访问生成器(Black-box Generators):如GPT系列,不允许内部结构更改或参数更新,专注于通过检索和增强过程来提升生成器的性能。
\subsection{检索集成方法}
检索集成方法(Retrieval-Integration Methods)是指在人工智能系统中,尤其是语言模型和信息检索系统中,将检索到的信息与模型生成的结果结合起来,以提高系统的性能和输出的相关性。这种方法特别适用于需要结合大量外部数据来提供准确和及时回答的场景。
输入层集成(Input-Layer Integration):将检索到的文档与原始查询结合,作为生成器的新输入。
输出层集成(Output-Layer Integration):在输出阶段将检索结果与生成结果结合起来,如kNN-LM通过插值两个下一个词的概率分布来进行预测。
中间层集成(Intermediate-Layer Integration):通过生成模型的内部层来整合检索结果,这种方法可能增加额外的复杂性,但也有望通过有效训练来提升生成模型的能力\cite{abeysinghe2023architectingintermediatelayersefficient}。
上下文集成:
在这种方法中,检索到的数据被用作上下文信息,提供给模型以帮助其更好地理解查询的上下文,并生成更准确的回答。

小型模型集成:
通过训练一个小型的语言模型来专门处理检索到的数据,然后将这个小型模型的输出作为输入提供给大型语言模型,以此来指导信息的整合。

微调集成:
使用外部领域特定的数据对预训练的大型语言模型进行微调,使其能够更好地处理特定领域的查询。 \cite{zhao2024retrievalaugmentedgenerationrag}
