\chapter{未来研究方向}

\section{多语言和多模态能力的提升}
RAG已经超越了最初基于文本的问题回答的限制,融入了各种各样的模态数据。
这种扩张催生了创新的多模式模式集成了不同领域的RAG概念的形象\cite{gao2024retrievalaugmentedgenerationlargelanguage}。RA-CM3\cite{yasunaga2023retrievalaugmentedmultimodallanguagemodeling}是多模态的先驱
检索和生成文本和图像的模型。BLIP-2\cite{li2023blip}利用了冻结图像编码器
用于高效视觉语言预训练的LLM,实现零样本图像到文本转换。“在你面前形象化
方法\cite{zhu2022visualize}使用图像生成来控制LM的文本生成,在开放式文本中显示出希望一代的任务。
音频和视频。GSS方法用于检索和缝合一起音频剪辑转换成机器翻译的数据
语音翻译数据\cite{zhao2022generating}。UEOP标志着端到端自动语音识别的重大进步
结合外部离线策略进行语音到文本转换\cite{chan2023using}。此外,基于KNN的注意力融合利用音频嵌入和语义相关的文本嵌入来完善ASR,从而加快域适应。
Vid2Seq用专门的时态增强语言模型
标记,便于预测事件边界和
统一输出序列中的文本描述\cite{yang2023vid2seq}。
代码。RBPS\cite{nashid2023retrieval}在小规模学习任务中表现出色,通过
检索与开发人员目标一致的代码示例
通过编码和频率分析。这种方法具有
在诸如测试断言生成和程序修复等任务中展示了有效性。对于结构化的知识,焦炭
方法\cite{li2023chain}首先提取与输入查询相关的事实
\
从知识图谱中,然后将这些事实集成为提示在输入中,提高知识图谱的性能问题的任务\cite{peng2024graphretrievalaugmentedgenerationsurvey}。

这篇文章\cite{li2024retrievalaugmentedgenerationlongcontext}对RAG和长上下文(Long-Context,简称LC)LLMs进行了全面比较,旨在利用两者的优势。
\section{外部知识的质量控制}

在考量大型语言模型(LLMs)的安全性和操纵性时,外部知识的质量控制至关重要。检索质量是检索增强型语言模型(RAG)系统有效性的根本,它直接影响到生成内容的相关性和准确性。然而,现有的检索方法常常面临挑战,比如数据中的噪声、不相关文档和碎片化信息,这些都可能干扰生成过程的质量。

关于数据中的噪声,分为有益噪声和有害噪声两大类。有益噪声,如语义噪声、数据类型噪声和非法句子噪声,可以提高模型的性能,增强模型对正确信息的识别能力。有害噪声,如反事实噪声、支持性噪声和拼写错误噪声,则会降低模型的性能。为了提高系统的抗噪声能力,通过系统地引入和管理噪声,可以提高模型的鲁棒性和适应性。

在检索结果不尽如人意时,模型可能会尝试生成不准确的回答,这增加了错误输出的风险。这种情况在查询含糊不清或缺乏足够上下文时尤为突出,使得检索模型难以找到相关的文档。例如,HyDE\cite{gao2022precisezeroshotdenseretrieval}通过生成一个能够捕捉查询核心的伪文档来解决这一问题。这种方法通过允许检索系统从非最优查询中检索到更多相关文档,从而提高了检索的准确性,尽管这可能会增加计算成本。未来的研究可以探索如何优化这一过程,以在不牺牲检索精度的情况下减少延迟。

对于信息的集成,复杂查询通常需要从多个文档中整合信息,但碎片化或相互矛盾的信息可能导致生成的答案不连贯或不完整。预检索和后检索技术在这里扮演了重要角色。通过提高检索粒度和采用实体级检索及重新排序技术,可以增强检索文档的连贯性。然而,许多后检索方法严重依赖于LLM API的调用,这可能导致成本过高。因此,研究更经济的替代方案,如将知识蒸馏到轻量级模型,可以提供更具可扩展性的解决方案,使高级检索策略在在线环境中更加实用。


\section{计算效率的优化}
RAG系统在处理大型数据集和实时应用时,系统效率仍然是一个显著的瓶颈。通过使用轻量级搜索方法、混合检索方法、可微分索引和优化的深度学习模型,可以提高系统性能和效率。而RALM带来了显著的计算开销,特别是在需要迭代推理的场景中。未来的研究可以专注于优化这些模型或开发检索精简技术,以减少传递到生成阶段的文档数量,同时不影响性能。

模块化工作流优化。RAG 系统的复杂性通常源于诸如分块策略、嵌入模型和重排序算法等组件之间的相互依赖。模块化设计是提高系统吞吐量的关键,它允许独立优化每个步骤,同时考虑跨组件的交互\cite{gao2024modularragtransformingrag}。先进的分块方法和混合搜索策略可以提供在最大程度上提高检索精度和速度之间的权衡\cite{huang2024surveyretrievalaugmentedtextgeneration}。

\section{RALM面临的安全问题}
LLM系统大大提高了工作效率,但是大语言模型的滥用会导致会导致负面的社会后果。
这种滥用包括学术欺诈、侵犯版权、网络攻击和利用软件漏洞
\cite{kuppamanipulating}。

TrojanRAG是一种针对检索增强型语言模型(RALM)的新型攻击方式,它利用了RALM的自然漏洞来注入联合后门,从而在各种通用攻击场景中操纵基于大型语言模型(LLMs)的API。这些攻击场景包括攻击者主动攻击、用户被动执行后门攻击以及后门式越狱攻击。TrojanRAG不仅能够在正常推理、可转移和CoT(思维链)中实现强大的后门激活,而且在日常查询中保持高可用性。这一点在正常查询中尤为重要,因为它意味着攻击者可以在不引起用户怀疑的情况下实施攻击\cite{cheng2024trojanragretrievalaugmentedgenerationbackdoor}。

也有研究\cite{du2024vulragenhancingllmbasedvulnerability}致力于利用LLMs和RAG框架来提高软件漏洞检测准确性。